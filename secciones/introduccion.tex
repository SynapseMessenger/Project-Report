\chapter{Introducción}

\section {Motivación}

La criptografía es la tecnología mas poderosa en existencia para preservar la privacidad. Un individuo es capaz de proteger información mediante algoritmos y métodos matemáticos de forma que ningún adversario sea capaz de descubrirla, sin importar la cantidad de esfuerzo (o capacidad de cómputo) que éste invierta. \\

La importancia de éste tipo de tecnologías radica en la protección de un individuo, sus acciones e identidad, en una realidad cada vez más conectada a Internet. \\

%\emph{ This is a quote }

\section {Tecnologías}
\subsection {Protocolo Signal}
\subsection {The Onion Router}
\subsection {Node.js}

Node.js es un interprete de JavaScript desarrollado sobre el motor V8 de Chrome. De forma que extiende el uso del lenguaje también en el lado del servidor. \\
Consta además, de uno de los ecosistemas de librerías más grandes en existencia. \\

Pese a disponer de una librería para el manejo de operaciones sobre HTTP, es también gracias a paquetes como Express o Hapi que Node.js se ha extendido en gran medida como herramienta para desarrollar servidores y aplicaciones web. \\
El modelo de Node.js esta basado en eventos y operaciones de E/S no bloqueantes, lo que lo hace eficiente en el tratamiento de peticiones concurrentes.

\subsection {Websockets}

Websocket es un protocolo que propociona un canal de comunicación bidireccional entre cliente y servidor sobre un mismo socket TCP. \\
El uso de sesiones de este tipo son especialmente útiles en aplicaciones donde es ventajoso que el servidor pueda enviar datos también al cliente. \\
Siendo este caso imposible en las formas de comunicación convencionales entre cliente y servidor como XHR (AJAX), donde es el cliente el encargado de iniciar cada intercambio de información. \\

El protocolo de websockets consta de soporte en las versiones más modernas de casi todos los navegadores. Sin embargo, puede no ser viable su uso en versiones más antiguas. \\

Para solventar este problema, se ha hecho uso de la librería Socket.io de Node.js. \\ 
Socket.io implementa una interfaz para la comunicación sobre websockets con un mecanismo para pivotar hacia otros modelos de comunicación en caso de que el medio no conste de soporte para éstos. \\
\subsection {Electron}
\subsection {React}
\subsection {Redux}