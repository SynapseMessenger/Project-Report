\chapter{Estado del arte}

Pese a que en los pasados años ha habido una creciente tendencia en la adopción de tecnologías de cifrado punto a punto en las aplicaciones de mensajería comerciales, ésto no significa directamente que los usuarios estén mejor protegidos o que los datos de las conversaciones privadas se traten de una forma más segura. \\

Esto es debido a que aunque muchos servicios tienen la opción de activar el cifrado punto a punto, impidiendo al servidor intermedio, es decir, al proveedor del servicio (Faceook, Google, etc), interceptar y analizar el contenido, la mayoría de éstos no lo tienen como opción activada por defecto. \\

Por lo tanto, el usuario ha de buscar y activar expresamente las funciones relacionadas con aumentar la seguridad, generalmente como la opción de ''Conversaciones secretas'', para que estas características se hagan efectivas. \\

A continuación se ha realizado un análisis de las aplicaciones más populares de mensajería y una consideración de sus características de seguridad, así como de la usabilidad o facilidad de acceso de los usuarios a las mismas. \\

Se ha dedicado un apartado individualmente a cada una de las aplicaciones más relevantes, haciendo hincapié en los pros y contras que presentan respecto a sus competidores. \\

\pagebreak

\section {Comparativa}

\begin{table}[h]
	\centering
	\label{comparativa-applicationes-mensajeria}
	\resizebox{\textwidth}{!}{%
		\begin{tabular}{|c|c|c|c|c|}
			\hline
			& \textbf{WhatsApp} & \textbf{Facebook Messenger} & \textbf{Telegram} & \textbf{Signal} \\ \hline
			Cifrado punto a punto                                                                                                 & Sí                & Sí                          & Sí                & Sí              \\ \hline
			\begin{tabular}[c]{@{}c@{}}Cifrado punto a punto \\ activado por defecto\end{tabular}                                 & Sí                & No                          & No                & Sí              \\ \hline
			\begin{tabular}[c]{@{}c@{}}Implicado en compartir\\ datos con agencias de inteligencia\end{tabular} & Sí                & Sí                          & No                & No              \\ \hline
			\begin{tabular}[c]{@{}c@{}}Compañía colecciona \\ datos de usuarios\end{tabular}                                      & Sí                & Sí                          & Sí                & No              \\ \hline
			Totalmente código abierto                                                                                             & No                & No                          & No                & Sí              \\ \hline
			\begin{tabular}[c]{@{}c@{}}Claves de cifrado generadas\\  en el dispositivo\end{tabular}                              & Sí                & Sí                          & Sí                & Sí              \\ \hline
			\begin{tabular}[c]{@{}c@{}}Mensajes cifrados \\ con claves individuales\end{tabular}                                  & Sí                & Sí                          & No                & Sí              \\ \hline
			\begin{tabular}[c]{@{}c@{}}Se almacenan tiempos de envío\\ y/o IPs\end{tabular}                                       & Sí                & Sí                          & Sí                & No              \\ \hline
			\begin{tabular}[c]{@{}c@{}}Reciente auditoría \\ de código y/o protocolo\end{tabular}                                 & No                & No                          & Sí                & Sí              \\ \hline
			Posibilidad de registro anónimo                                                                                       & No                & No                          & No                & No              \\ \hline
			\begin{tabular}[c]{@{}c@{}}Compañía puede leer\\ contenido de mensajes\end{tabular}                                   & No                & Sí                          & Sí                & No              \\ \hline
		\end{tabular}%
	}
\end{table}

\subsubsection {WhatsApp}

Con aproximadamente mil trescientos millones de usuarios diarios (Julio 2017), WhatsApp cuenta con la corona en el mercado de aplicaciones de mensajería. \\

Fue fundada en 2009, por Jan Koum, y gracias a la adopción de internet en los dispositivos móviles y la ubiquidad de los mismos, ha crecido exponencialmente desde entonces. \\
Su rápida adopción fue también debido a la enorme ventaja contra los SMS, que contaban con un límite de caracteres y su coste de uso era mucho mayor. \\

En febrero de 2014, Facebook anunció su compra, y fue adquirida finalmente por 21800 millones de dólares. \\

Gracias a una colaboración con Open Whispers, a día de hoy cuenta con la integración del protocolo Signal en todas sus comunicaciones; mensajes de texto, voz y videollamadas. \\ Todas sus formas de comunicación vienen cifradas punto a punto por defecto y de forma transparente al usuario. \\
A efectos prácticos, WhatsApp protege a sus usuarios de espías intermedios y tampoco son capaces de leer el contenido de las comunicaciones en sus propios servidores. \\

Sin embargo, WhatsApp sí que almacena los tiempos de envío de los mensajes, así como identidad del emisor y receptor de los mismos; entre otra tanta información catalogada como metadatos. \\

Dichos metadatos pueden definirse de una forma bastante certera como la información referente a actividad. Es decir, no es contenido propiamente dicho, pero sí toda la información respecto a la naturaleza del mismo. \\
Por tanto, en esta categoría entran aspectos como: Información respecto al dispositivo, posición GPS, contactos, horas de conexión, conexiones inalámbricas, duración de las comunicaciones, tamaño de los paquetes enviados, etc.

Por otro lado, Facebook fue mencionado en los documentos filtrados por Edward Snowden en 2013 como una de las compañías que cedía el acceso a los datos de sus usuarios a la NSA (Agencia de seguridad nacional) de EEUU. \\

En conclusión, WhatsApp se presenta como una herramienta de comunicación que cuenta con la tecnología y las herramientas adecuadas pero implementadas de forma que no se cubren completamente los aspectos necesarios para garantizar una privacidad absoluta de sus usuarios, así como la protección de los mismos frente a un escrutinio por parte de un adversario como una entidad gubernamental. \\

\subsubsection {Facebook Messenger}

Facebook Messenger es la aplicación de mensajería que acompaña a la aplicación móvil de la red social. \\
Fue lanzada en 2004 bajo el nombre de Facebook Chat, y desde entonces ha sido adoptada por una enorme cantidad de usuarios. Uno de los posibles motivos de esta acogida puede ser el hecho de que Facebook la ha convertido en un requisito para establecer comunicaciones en la red social, de forma que es imposible continuar con la misma experiencia de uso en Facebook sin descargarla e instalarla. \\

Como se ha mencionado anteriormente en la tabla comparativa, Facebook Messenger dispone de cifrado punto a punto en forma de chats secretos, que no están habilitados por defecto. \\
En su uso normal, la aplicación solamente cifra el tráfico entre los usuarios y sus servidores. Es decir, la empresa es capaz de leer el contenido de las comunicaciones (tanto texto como contenido multimedia). \\

Facebook Messenger es un claro ejemplo de la adopción de una tecnología que debería ser ubicua en todas las comunicaciones modernas, pero implementada con lo que parece ser una motivación puramente comercial, ya que la enorme mayoría de usuarios no conoce la existencia de ésta característica; y los que la conocen, aún son pocos los que la utilizan. \\

\subsubsection {Telegram}

Telegram es una aplicación de mensajería fundada y desarrollada desde 2013 por los hermanos Nikolai y Pavel Durov. Pavel Durov es un conocido emprendedor y millonario de origen ruso, famoso por ser el fundador de la red social VK. \\ 

Nikolai, por su parte, fue CTO de la anterior mencionada red social, y cuenta con una extensa formación académica tanto en matemáticas como en informática; ha obtenido, además, diversas medallas y condecoraciones en olimpiadas y competiciones en ambos campos. \\

El protocolo de Telegram, MTProto, fue desarrollado desde cero por Nikolai, lo cual ha ocasionado diversas dudas respecto a su seguridad y fiabilidad para un uso en producción en la escala de Telegram. \\

Por otro lado, pese a disponer de la opción de chats secretos cifrados punto a punto, éstos no se encuentran activados por defecto. En consecuencia, las conversaciones pueden ser leídas y el contenido visualizado en los servidores de la compañía sin ningún impedimento. \\

La función de chat privados hace uso de un cifrado punto a punto donde las claves generadas en el dispositivo son reutilizadas múltiples veces, por lo que se pierde la característica de ''forward secrecy''. \\ Si se logra acceder a las claves de cifrado, todos los mensajes serán accesibles para un atacante. \\
