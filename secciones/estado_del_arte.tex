\chapter{Estado del arte}

Pese a que en los pasados años ha habido una creciente tendencia en la adopción de tecnologías de cifrado punto a punto en las aplicaciones de mensajería comerciales, ésto no significa directamente que los usuarios estén mejor protegidos o que los datos de las conversaciones privadas se traten de una forma más segura. \\

Esto es debido a que aunque muchos servicios tienen la opción de activar el cifrado punto a punto, impidiendo al servidor intermedio, es decir, al proveedor del servicio (Faceook, Google, etc), interceptar y analizar el contenido, la mayoría de éstos no lo tienen como opción activada por defecto. \\

Por lo tanto, el usuario ha de buscar y activar expresamente las funciones relacionadas con aumentar la seguridad, generalmente como la opción de "Conversaciones secretas", para que estas características se hagan efectivas. \\

A continuación se ha realizado un análisis de las aplicaciones más populares de mensajería y una consideración de sus características de seguridad, así como de la usabilidad o facilidad de acceso de los usuarios a las mismas. \\

